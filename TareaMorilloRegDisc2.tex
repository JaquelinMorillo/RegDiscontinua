% Options for packages loaded elsewhere
\PassOptionsToPackage{unicode}{hyperref}
\PassOptionsToPackage{hyphens}{url}
%
\documentclass[
]{article}
\usepackage{amsmath,amssymb}
\usepackage{lmodern}
\usepackage{iftex}
\ifPDFTeX
  \usepackage[T1]{fontenc}
  \usepackage[utf8]{inputenc}
  \usepackage{textcomp} % provide euro and other symbols
\else % if luatex or xetex
  \usepackage{unicode-math}
  \defaultfontfeatures{Scale=MatchLowercase}
  \defaultfontfeatures[\rmfamily]{Ligatures=TeX,Scale=1}
\fi
% Use upquote if available, for straight quotes in verbatim environments
\IfFileExists{upquote.sty}{\usepackage{upquote}}{}
\IfFileExists{microtype.sty}{% use microtype if available
  \usepackage[]{microtype}
  \UseMicrotypeSet[protrusion]{basicmath} % disable protrusion for tt fonts
}{}
\makeatletter
\@ifundefined{KOMAClassName}{% if non-KOMA class
  \IfFileExists{parskip.sty}{%
    \usepackage{parskip}
  }{% else
    \setlength{\parindent}{0pt}
    \setlength{\parskip}{6pt plus 2pt minus 1pt}}
}{% if KOMA class
  \KOMAoptions{parskip=half}}
\makeatother
\usepackage{xcolor}
\usepackage[margin=1in]{geometry}
\usepackage{graphicx}
\makeatletter
\def\maxwidth{\ifdim\Gin@nat@width>\linewidth\linewidth\else\Gin@nat@width\fi}
\def\maxheight{\ifdim\Gin@nat@height>\textheight\textheight\else\Gin@nat@height\fi}
\makeatother
% Scale images if necessary, so that they will not overflow the page
% margins by default, and it is still possible to overwrite the defaults
% using explicit options in \includegraphics[width, height, ...]{}
\setkeys{Gin}{width=\maxwidth,height=\maxheight,keepaspectratio}
% Set default figure placement to htbp
\makeatletter
\def\fps@figure{htbp}
\makeatother
\setlength{\emergencystretch}{3em} % prevent overfull lines
\providecommand{\tightlist}{%
  \setlength{\itemsep}{0pt}\setlength{\parskip}{0pt}}
\setcounter{secnumdepth}{-\maxdimen} % remove section numbering
\usepackage{booktabs}
\usepackage{siunitx}

  \newcolumntype{d}{S[
    input-open-uncertainty=,
    input-close-uncertainty=,
    parse-numbers = false,
    table-align-text-pre=false,
    table-align-text-post=false
  ]}
  
\usepackage{longtable}
\usepackage{array}
\usepackage{multirow}
\usepackage{wrapfig}
\usepackage{float}
\usepackage{colortbl}
\usepackage{pdflscape}
\usepackage{tabu}
\usepackage{threeparttable}
\usepackage{threeparttablex}
\usepackage[normalem]{ulem}
\usepackage{makecell}
\usepackage{xcolor}
\ifLuaTeX
  \usepackage{selnolig}  % disable illegal ligatures
\fi
\IfFileExists{bookmark.sty}{\usepackage{bookmark}}{\usepackage{hyperref}}
\IfFileExists{xurl.sty}{\usepackage{xurl}}{} % add URL line breaks if available
\urlstyle{same} % disable monospaced font for URLs
\hypersetup{
  pdftitle={Tarea Regresión Discontinua},
  pdfauthor={Jaquelin Morillo},
  hidelinks,
  pdfcreator={LaTeX via pandoc}}

\title{Tarea Regresión Discontinua}
\author{Jaquelin Morillo}
\date{2022-12-15}

\begin{document}
\maketitle

\hypertarget{introducciuxf3n}{%
\subsubsection{Introducción}\label{introducciuxf3n}}

Dentro de regresiones discontinuas, hay un tipo particular de estrategia
y diseño metodológico que se ha hecho popular: el diseño de elecciones
ajustadas. Esencialmente, este diseño explota una característica de las
democracias americanas según la cual los ganadores de las contiendas
políticas se declaran cuando un candidato obtiene el porcentaje mínimo
necesario de votos. En la medida en que las elecciones muy reñidas
representan asignaciones exógenas de la victoria de un partido, podemos
utilizar estas elecciones reñidas para identificar el efecto causal del
ganador en una serie de resultados. También podemos poner a prueba
teorías de economía política que, de otro modo, serían casi imposibles
de evaluar.

¿Son los políticos o los votantes quienes eligen las políticas? La gran
pregunta que motiva a Lee et al.~(2004) tiene que ver con si los
votantes afectan a la política y de qué manera. Hay dos visiones
fundamentalmente diferentes del papel de las elecciones en una
democracia representativa: la teoría de la convergencia y la teoría de
la divergencia. La teoría de la convergencia afirma que la ideología
heterogénea de los votantes obliga a cada candidato a moderar su
posición. La teoría de la divergencia, por su parte, sostiene que cuando
los políticos partidistas no pueden comprometerse de forma creíble con
determinadas políticas, la convergencia se ve socavada y el resultado
puede ser una ``divergencia'' política total. La divergencia se produce
cuando el candidato ganador, tras tomar posesión de su cargo, se limita
a aplicar su política preferida. En este caso extremo, los votantes son
incapaces de obligar a los candidatos a alcanzar ningún tipo de
compromiso político, lo que se traduce en dos candidatos opuestos que
eligen políticas muy diferentes en distintos escenarios contrafactuales
de victoria.

\hypertarget{modelo}{%
\subsubsection{Modelo}\label{modelo}}

\(R\) y \(D\) son candidatos compitiendo por un escaño en el Congreso.
El espacio político es una sola dimensión en la que las preferencias
políticas de \(R\) y \(D\) en un periodo son funciones de pérdida
cuadráticas, \(u(l)\) y \(v(l)\), y \(l\) es la variable política.

Cada jugador tiene un punto de felicidad, que es su posición preferida
en la gama unidimensional de políticas. Para los demócratas, es
\(l* =c(>0)\), y para los republicanos es \(l* = 0\).

El posible resultado de la votación nominal del candidato tras unas
elecciones es:

\[RC_t= D_t x_t+(1-D_t)y_t\]

Donde \(D_t\) indica si un demócrata ganó las elecciones. Es decir, sólo
se observa la política del candidato ganador. A contuniación se expresan
las ecuaciones de regresión derivadas de la expresión anterior:

\[RC_t= \alpha_0 +\pi_0 P^{*}_t+\pi_1 D_t+\varepsilon_t \]
\[RC_{t+1}= \beta_0 +\pi_0 P^{*}_{t+1}+\pi_1 D_{t+1}+\varepsilon_{t+1}\]

Donde \(\alpha_0\) y \(\beta_0\) son constantes y \(P^{*}\) representa
la popularidad subyacente del Partido Demócrata, o dicho de otro modo,
la probabilidad de que \(D\) ganara si la política elegida \(x\)
igualara el punto \(c\) de felicidad del demócrata.

Es importante destacar que la asignación aleatoria de \(D_t\) es
crucial. Porque sin ella, esta ecuación reflejaría \(\pi_1\) y la
selección (es decir, los distritos demócratas tienen puntos de felicidad
más liberales). Así que los autores intentan aleatorizar \(D_t\)
utilizando un RDD. Para efectuar la asignacion aleatoria, los autores
utilizan una variación posiblemente exógena en las victorias demócratas
para comprobar si la convergencia o la divergencia son correctas. Si la
convergencia es cierta, entonces los republicanos y los demócratas que
apenas ganaron deberían votar casi idénticamente, mientras que si la
divergencia es cierta, deberían votar de forma diferente en los márgenes
de una carrera reñida. ``En los márgenes de una carrera reñida'' es
crucial porque la idea es que es en los márgenes de una carrera reñida
donde la distribución de las preferencias de los votantes es
practicamnerte, la misma. Si las preferencias de los votantes son las
mismas, pero las políticas divergen en el límite, esto sugiere que son
los políticos, y no los votantes, los que dirigen la elaboración de las
políticas.

Pare replicar los resultados de los autores se utilizán regresiones
limitadas a la ventana alrededor del punto de corte para estimar el
efecto. Se trata de regresiones locales en el sentido de que utilizan
datos cercanos al punto de corte. Sólo se utilizan observaciones entre
0,48 y 0,52 votos. Por tanto, esta regresión estima el coeficiente en
torno al punto de corte.

\begin{table}
\centering
\begin{tabular}[t]{lccc}
\toprule
  & Model 1 & Model 2 & Model 3\\
\midrule
(Intercept) & \num{31.196}*** & \num{18.747}*** & \num{0.242}***\\
lagdemocrat & \num{21.284}*** &  & \num{0.484}***\\
democrat &  & \num{47.706}*** & \\
\midrule
Num.Obs. & \num{915} & \num{915} & \num{915}\\
R2 & \num{0.115} & \num{0.578} & \num{0.235}\\
R2 Adj. & \num{0.114} & \num{0.578} & \num{0.234}\\
RMSE & \num{29.49} & \num{20.36} & \num{0.44}\\
Std.Errors & by: id & by: id & by: id\\
\bottomrule
\end{tabular}
\end{table}

Cuando se utilizan todos los datos, obtenemos efectos algo diferentes.
El efecto sobre los resultados futuros del Americans for Democratic
Action (ADA) aumenta en 10 puntos, pero el efecto contemporáneo
disminuye. Sin embargo, el efecto sobre la ocupación del cargo aumenta
considerablemente. Así que aquí vemos que simplemente ejecutando la
regresión se obtienen estimaciones diferentes cuando incluimos datos
alejados del propio punto de corte.

\begin{table}
\centering
\begin{tabular}[t]{lccc}
\toprule
  & Model 1 & Model 2 & Model 3\\
\midrule
(Intercept) & \num{23.539}*** & \num{17.576}*** & \num{0.120}***\\
lagdemocrat & \num{31.506}*** &  & \num{0.818}***\\
democrat &  & \num{40.763}*** & \\
\midrule
Num.Obs. & \num{13588} & \num{13588} & \num{13588}\\
R2 & \num{0.227} & \num{0.376} & \num{0.676}\\
R2 Adj. & \num{0.227} & \num{0.376} & \num{0.676}\\
RMSE & \num{28.69} & \num{25.78} & \num{0.28}\\
Std.Errors & by: id & by: id & by: id\\
\bottomrule
\end{tabular}
\end{table}

Sin embargo, ninguna de estas regresiones incluye controles para la
running variable. Tampoco se utiliza el recentrado de la variable de
ejecución. Para incorporar estos dos puntos, simplemente restaremos 0,5
a la variable de funcionamiento, de forma que los valores de 0 sean
aquellos en los que el porcentaje de votos sea igual a 0,5, los valores
negativos sean los porcentajes de votos demócratas inferiores a 0,5 y
los valores positivos sean los porcentajes de votos demócratas
superiores a 0,5.

\begin{table}
\centering
\begin{tabular}[t]{lccc}
\toprule
  & Model 1 & Model 2 & Model 3\\
\midrule
(Intercept) & \num{22.883}*** & \num{11.034}*** & \num{0.212}***\\
lagdemocrat & \num{33.451}*** &  & \num{0.552}***\\
demvoteshare\_c & \num{-5.626}** & \num{-48.938}*** & \num{0.773}***\\
democrat &  & \num{58.502}*** & \\
\midrule
Num.Obs. & \num{13577} & \num{13577} & \num{13577}\\
R2 & \num{0.227} & \num{0.424} & \num{0.735}\\
R2 Adj. & \num{0.227} & \num{0.424} & \num{0.735}\\
RMSE & \num{28.68} & \num{24.76} & \num{0.25}\\
Std.Errors & by: id & by: id & by: id\\
\bottomrule
\end{tabular}
\end{table}

Es habitual permitir que la variable en curso varíe a ambos lados de la
discontinuidad, Para implementarlo necesitamos que haya una línea de
regresión a cada lado, lo que significa necesariamente que tenemos dos
líneas a la izquierda y a la derecha de la discontinuidad. Para ello,
necesitamos una interacción, específicamente, entre la variable de
ejecución con la variable de tratamiento.

\begin{table}
\centering
\begin{tabular}[t]{lccc}
\toprule
  & Model 1 & Model 2 & Model 3\\
\midrule
(Intercept) & \num{31.435}*** & \num{16.816}*** & \num{0.287}***\\
lagdemocrat & \num{30.508}*** &  & \num{0.526}***\\
demvoteshare\_c & \num{66.042}*** & \num{-5.683}* & \num{1.403}***\\
lagdemocrat × demvoteshare\_c & \num{-96.475}*** &  & \num{-0.849}***\\
democrat &  & \num{55.431}*** & \\
democrat × demvoteshare\_c &  & \num{-55.152}*** & \\
\midrule
Num.Obs. & \num{13577} & \num{13577} & \num{13577}\\
R2 & \num{0.267} & \num{0.434} & \num{0.749}\\
R2 Adj. & \num{0.267} & \num{0.434} & \num{0.749}\\
RMSE & \num{27.94} & \num{24.54} & \num{0.25}\\
Std.Errors & by: id & by: id & by: id\\
\bottomrule
\end{tabular}
\end{table}

Por último, se estimará el modelo con una cuadrática. La inclusión de la
cuadrática hace que el efecto estimado de una victoria democrática sobre
el voto futuro disminuya considerablemente:

\begin{table}
\centering
\begin{tabular}[t]{lccc}
\toprule
  & Model 1 & Model 2 & Model 3\\
\midrule
(Intercept) & \num{33.547}*** & \num{15.606}*** & \num{0.330}***\\
lagdemocrat & \num{13.030}*** &  & \num{0.322}***\\
demvoteshare\_c & \num{134.977}*** & \num{-23.850}*** & \num{2.798}***\\
demvoteshare\_sq & \num{212.127}*** & \num{-41.729}** & \num{4.294}***\\
lagdemocrat × demvoteshare\_c & \num{57.055}*** &  & \num{0.091}\\
lagdemocrat × demvoteshare\_sq & \num{-641.851}*** &  & \num{-8.804}***\\
democrat &  & \num{44.402}*** & \\
democrat × demvoteshare\_c &  & \num{111.896}*** & \\
democrat × demvoteshare\_sq &  & \num{-229.954}*** & \\
\midrule
Num.Obs. & \num{13577} & \num{13577} & \num{13577}\\
R2 & \num{0.371} & \num{0.456} & \num{0.822}\\
R2 Adj. & \num{0.370} & \num{0.456} & \num{0.822}\\
RMSE & \num{25.89} & \num{24.07} & \num{0.21}\\
Std.Errors & by: id & by: id & by: id\\
\bottomrule
\end{tabular}
\end{table}

Esto sugiere que existen fuertes valores atípicos en los datos que están
causando que la distancia en \(c_0\) se extienda más ampliamente. Así
que una solución natural es limitar de nuevo nuestro análisis a una
ventana más pequeña. Lo que esto hace es descartar las observaciones
alejadas de \(c_0\) y omitir la influencia de los valores atípicos de
nuestra estimación en el punto de corte. Esta vez utilizaremos +/-
-0,05. Cuando limitamos nuestro análisis a 0,05 alrededor del punto de
corte, se utilizan más observaciones lejos del punto de corte que las
que se utilizaron en nuestro análisis inicial. Por eso sólo tenemos
2.441 observaciones para el análisis, frente a las 915 que teníamos en
nuestro análisis original.

\begin{table}
\centering
\begin{tabular}[t]{lccc}
\toprule
  & Model 1 & Model 2 & Model 3\\
\midrule
(Intercept) & \num{37.121}*** & \num{21.437}*** & \num{0.418}***\\
lagdemocrat & \num{7.347}*** &  & \num{0.167}***\\
demvoteshare\_c & \num{830.925}*** & \num{450.846}** & \num{15.699}***\\
demvoteshare\_sq & \num{5333.335}*** & \num{7878.904}** & \num{91.607}***\\
lagdemocrat × demvoteshare\_c & \num{-156.876}*** &  & \num{0.125}\\
lagdemocrat × demvoteshare\_sq & \num{-10116.678}*** &  & \num{-188.329}***\\
democrat &  & \num{45.191}*** & \\
democrat × demvoteshare\_c &  & \num{-688.343}** & \\
democrat × demvoteshare\_sq &  & \num{-3887.820} & \\
\midrule
Num.Obs. & \num{2387} & \num{2387} & \num{2387}\\
R2 & \num{0.445} & \num{0.563} & \num{0.774}\\
R2 Adj. & \num{0.444} & \num{0.562} & \num{0.774}\\
RMSE & \num{23.50} & \num{20.86} & \num{0.24}\\
Std.Errors & by: id & by: id & by: id\\
\bottomrule
\end{tabular}
\end{table}

Los métodos no paramétricos significan muchas cosas diferentes para
distintas personas en estadística, pero en contextos de RDD, la idea es
estimar un modelo que no asuma una forma funcional para la relación
entre la variable de resultado (Y) y la variable de ejecución (X). A
continuación se presentan gráficamente distintas opciones de ajuste,
lineal, cuadrático y lowess:

\includegraphics{TareaMorilloRegDisc2_files/figure-latex/unnamed-chunk-7-1.pdf}
\includegraphics{TareaMorilloRegDisc2_files/figure-latex/unnamed-chunk-7-2.pdf}
\includegraphics{TareaMorilloRegDisc2_files/figure-latex/unnamed-chunk-7-3.pdf}

Hahn, Todd y Klaauw (2001) han demostrado que la estimación kernel
unilateral, como lowess, puede tener propiedades deficientes porque el
punto de interés se encuentra en el límite (es decir, la
discontinuidad). Esto se llama el ``problema del límite''. Proponen
utilizar en su lugar regresiones lineales locales no paramétricas. En
estas regresiones, se da más peso a las observaciones del centro.

También puede estimar regresiones polinómicas locales ponderadas por
kernel, una regresión ponderada restringida a una ventana como las que
se han realizado previamnete, donde el kernel elegido proporciona los
pesos.

\includegraphics{TareaMorilloRegDisc2_files/figure-latex/unnamed-chunk-8-1.pdf}

Debido a que el supuesto de continuidad implica específicamente
funciones de expectativas condicionales continuas de los resultados
potenciales a lo largo de la línea de corte, por lo tanto, no se puede
probar. Eso es correcto, es un supuesto no comprobable. Pero, lo que
podemos hacer es comprobar si hay cambios en las funciones de
expectativa condicional para otras covariables exógenas que no pueden o
no deben cambiar como resultado del corte. Luego, la importancia de la
selección del ancho de banda, o ventana, para estimar el efecto causal
utilizando este método, así como la importancia de la selección de la
longitud del polinomio. A la hora de elegir el ancho de banda, siempre
hay un equilibrio entre el sesgo y la varianza: cuanto más corta es la
ventana, menor es el sesgo, pero al tener menos datos, aumenta la
varianza de la estimación.

\begin{verbatim}
## Sharp RD estimates using local polynomial regression.
## 
## Number of Obs.                13577
## BW type                       mserd
## Kernel                   Triangular
## VCE method                       NN
## 
## Number of Obs.                 5480         8097
## Eff. Number of Obs.            2112         1893
## Order est. (p)                    1            1
## Order bias  (q)                   2            2
## BW est. (h)                   0.086        0.086
## BW bias (b)                   0.141        0.141
## rho (h/b)                     0.609        0.609
## Unique Obs.                    2770         3351
## 
## =============================================================================
##         Method     Coef. Std. Err.         z     P>|z|      [ 95% C.I. ]       
## =============================================================================
##   Conventional    46.491     1.241    37.477     0.000    [44.060 , 48.923]    
##         Robust         -         -    31.425     0.000    [43.293 , 49.052]    
## =============================================================================
\end{verbatim}

Lo ideal es utilizar este tipo de métodos cuando se dispone de un gran
número de observaciones en la muestra, de forma que se tenga un número
considerable de observaciones en la discontinuidad. En ese caso, debería
haber cierta armonía entre los resultados. Si no es así, es posible que
no tenga suficiente potencia para detectar este efecto.

Por último, se examina la implementación de la prueba de densidad de
McCrary, utilizando la estimación de densidad polinómica local
(Cattaneo, Jansson y Ma 2019) y no se obtiene evidencia de que haya
habido manipulación en la variable de ejecución en el punto de corte.

\includegraphics{TareaMorilloRegDisc2_files/figure-latex/unnamed-chunk-10-1.pdf}

\begin{verbatim}
## [1] 0.5925909
\end{verbatim}

\includegraphics{TareaMorilloRegDisc2_files/figure-latex/unnamed-chunk-10-2.pdf}

\begin{verbatim}
## $Estl
## Call: lpdensity
## 
## Sample size                                      5480
## Polynomial order for point estimation    (p=)    2
## Order of derivative estimated            (v=)    1
## Polynomial order for confidence interval (q=)    3
## Kernel function                                  triangular
## Scaling factor                                   0.40357984678845
## Bandwidth method                                 user provided
## 
## Use summary(...) to show estimates.
## 
## $Estr
## Call: lpdensity
## 
## Sample size                                      8097
## Polynomial order for point estimation    (p=)    2
## Order of derivative estimated            (v=)    1
## Polynomial order for confidence interval (q=)    3
## Kernel function                                  triangular
## Scaling factor                                   0.596346493812611
## Bandwidth method                                 user provided
## 
## Use summary(...) to show estimates.
## 
## $Estplot
\end{verbatim}

\includegraphics{TareaMorilloRegDisc2_files/figure-latex/unnamed-chunk-10-3.pdf}

\end{document}
